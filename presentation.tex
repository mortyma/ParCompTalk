\documentclass{beamer}
\let\Tiny=\tiny %to avoid warnings related to font size and beamer 
\usetheme{Amsterdam}
\usecolortheme{dolphin}

\usepackage{comment}

\usepackage[style=numeric-comp,backend=bibtex]{biblatex}
\addbibresource{sources}

%reduce font size of footnotes
\let\oldfootnotesize\footnotesize
\renewcommand*{\footnotesize}{\oldfootnotesize\tiny}

\newcommand\fR[1]{\textcolor{red!80!black}{\textbf{#1}}}


%---------------------------------------------------------------------
%Front matter
%---------------------------------------------------------------------
\title{INSPIRE}
\subtitle{The Insieme Parallel Intermediate Representation}
\author{Martin Kalany}
\institute
{
  Graduate student in Computer Science\\
  Vienna University of Technology\\
}
\date{\today}

\begin{document}
\maketitle

\begin{frame}
\frametitle{Some standards for parallel programming}
\begin{itemize}
\item \fR{MPI} message passing, distributed memory
\item \fR{OpenMP} shared memory
\item \fR{OpenCL} across heterogenous platforms
\end{itemize}
\end{frame}

\begin{frame}
\frametitle{The problem}

\fR{Compiler view:} just another library used by a sequential host language

\bigskip\pause
\fR{Parallel control flow} remains \fR{hidden} in library calls embedded in sequential IR\footnote{IR: intermediate representation }

\bigskip\pause
Sequential IR \fR{cannot represent parallelism} appropriately

\bigskip\pause
$\Rightarrow$ \fR{Optimizations} of parallel constructs \fR{not possible} for compiler. 
\end{frame}

\begin{frame}
\frametitle{The idea}
\fR{INSPIRE} is a
\begin{itemize}
\item unified,
\item parallel,
\item high-level IR

modelling parallel constructs explicitly.

\bigskip\pause
$\Rightarrow$ Make parallel control flow available to compiler
\end{itemize}

\end{frame}

\begin{frame}
\cite{JordanPTKF13}
\end{frame}

%---------------------------------------------------------------------
%bibliography
%---------------------------------------------------------------------
\begin{frame}[allowframebreaks]
\frametitle<presentation>{Literature}    
\printbibliography
\end{frame} 	 
\end{document}